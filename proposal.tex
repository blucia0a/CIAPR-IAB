\newcount\draft\draft=1 % set to 0 for submission or publication
\newcount\cameraready\cameraready=0

\documentclass[9pt,parskip=full]{article}
\usepackage{fullpage}
\usepackage{graphicx}
\DeclareGraphicsExtensions{.pdf,.jpg,.png}
\graphicspath{{./figs/}}

\usepackage{listings}
\lstset{
    language=C,
    emphstyle=\bfseries,
    basicstyle=\ttfamily\small,
    aboveskip=1mm plus 1mm minus 1mm,
    belowskip=1mm plus 1mm minus 1mm,
    mathescape=true,
    xleftmargin=\parindent,
}
\newcommand{\lil}{\lstinline}


\title{\fontfamily{\sfdefault}{\em netcat}: Improvising with Computer Networks}
\author{\fontfamily{\sfdefault}{\em netcat}David Balatero and Brandon Lucia}
%\authorinfo{Anonymous for Submission}{}
\date{}

\newcommand{\term}[1]{\emph{#1}}

\begin{document}

\special{papersize=8.5in,11in}
\setlength{\pdfpageheight}{\paperheight}
\setlength{\pdfpagewidth}{\paperwidth}

\maketitle

\abstract{

We have developed a system that examines arbitrary network communication
between arbitrary computers and chooses how to play music corresponding to
those communications in real time.  We identify several key attributes of
network communication, focusing on their origin and destination, the type of
message ({\em i.e.}, port), and the timing of communications.  Given a
particular communication between computers, our system makes improvisational
decisions based on the timing of messages to determine when notes are played,
based on the type of message to determine the voice of the note, and based on
the origin and destination to determine the pitch of the note. 
\\[12pt] 
Our system participates in improvisation in two ways:
\\[3pt]
\begin{enumerate}

\item{{\bf As an improviser:} the system passively listens to naturally occurring
network communication to make improvisational decisions.}

\item{{\bf As an instrument:} a person can create network traffic in real time,
by taking specific actions -- {\em e.g.}, browsing the Web, checking email,
using low-level network communication software, {\em etc.} -- that the system
uses to make improvisational decisions.\\[3pt]}

\end{enumerate}
Human improvisers can improvise with the system by playing it as an instrument
or by playing along with it using conventional instruments.  We have composed
an improvisational piece of music called {\em netcat} that uses our system. The
piece incorporates human improvisers on conventional instruments playing
alongside our system.  Our system improvises by generating improvised music
from both passive and human-curated network traffic. {\em netcat} illustrates
that computers, a pervasive form of technology, do, indeed, improvise. The
piece further explores humans' interactions with technology that
collaboratively create improvised music.    
\\[12pt] 
We perform {\em netcat} by using computers to send our system live Internet
traffic, letting our system passively observe naturally occurring communication
between multiple networked computers, and human performers playing cello,
synthesizer, and drums. We are prepared to travel to Prague, perform the piece,
and facilitate discussion on computers, technology and improvisation, and
musical human-computer interaction in general at IAB 2014.
}


%\bibliographystyle{abbrv}
%\bibliography{harvsim}



\end{document}

